%----------------------------------------------------------------------------------------
%	PÁGINA DE APRESENTAÇÃO AO LEITOR
%----------------------------------------------------------------------------------------

\begin{flushleft}
\Huge\textbf{Prezado(a) leitor(a),}
\end{flushleft}

Fico feliz e agradeço por você ter escolhido ler a Apostila de Introdução ao Jamovi. Espero que este recurso seja útil para você em sua jornada de aprendizado e aplicação de análises estatísticas com o Jamovi.

É importante mencionar também que a apostila está em constante atualização. À medida que novas versões forem lançadas, novos recursos, exemplos e aprimoramentos serão adicionados. Recomento que você visite regularmente o site do \href{https://www.balaiocientifico.com/jamovi/apostila-de-jamovi/?utm_source=apostila_jamovi&utm_medium=pdf} {Balaio Científico} para obter a versão mais atualizada da apostila. Dessa forma, você poderá se beneficiar de novos conteúdos e melhorias à medida que são disponibilizados.

Além disso, incentivo você a compartilhar e usar este material com outras pessoas interessadas em aprender sobre análise estatística com o Jamovi. Acredito no poder do conhecimento compartilhado e na colaboração para promover um aprendizado mais amplo e significativo.

Se você deseja colaborar com esse projeto, você pode acessar o repositório dessa apostila no \href{https://github.com/balaio-cientifico/apostila_jamovi}{Github}. Se quiser entrar em contato comigo você pode me enviar um e-mail, ou entrar em contato comigo por uma das redes sociais listadas abaixo.

Mais uma vez, agradeço sua leitura e interesse pela Apostila do Jamovi. Espero que ela seja útil para você em seus estudos e projetos. Se você tiver alguma dúvida, sugestão ou feedback, não hesite em entrar em contato comigo. Estou sempre buscando melhorar e fornecer recursos de alta qualidade para a comunidade.

Aproveite o material e bons estudos!

\vfill

% Seção de redes sociais
\noindent \textbf{Me Siga nas redes:}

\noindent \faTwitter{} \href{https://twitter.com/balaioci}{balaioci} \\
\faFacebook{} \href{https://www.facebook.com/balaiocientifico/}{balaiocientifico} \\
\faYoutube{} \href{https://www.youtube.com/@BalaioCientifico}{@BalaioCientifico} \\
\faInstagram{} \href{https://www.instagram.com/balaiocientifico/}{balaiocientifico} \\
\faEnvelope{} \href{mailto:contato@balaiocientifico.com}{erickfaria@balaiocientifico.com}\\

% Seção de livros
\noindent \textbf{Conheça meus livros:}

\noindent \faAmazon{} \href{https://www.amazon.com.br/Erick-Faria/e/B09SZPGBZL/}{Amazon} \\

\vfill
\centerline{Atualizado em: {\noindent \today}}

\newpage

%----------------------------------------------------------------------------------------
%	PÁGINA DE APOIO AO BALAIO CIENTÍFICO
%----------------------------------------------------------------------------------------

\begin{center}
\Large\textbf{Apoie o Balaio Científico}
\end{center}

\vspace{2mm}

Caro(a) leitor(a),

No Balaio Científico, acreditamos que o conhecimento deve ser democrático. Por isso, a maior parte do nosso material — aulas, tutoriais e ferramentas — é disponibilizada gratuitamente para todos(as). Produzir material educativo de qualidade, no entanto, demanda tempo e recursos. É por isso que criei um programa de membros, onde você pode apoiar este trabalho e, em troca, receber benefícios exclusivos.

% Por que se tornar um membro
\vspace{1mm}
\begin{center}
\textbf{Por que se tornar um membro?}
\end{center}

\vspace{1mm}
\noindent\begin{tabular}{p{1cm}p{13.5cm}}
\textcolor{red}{\faQuestionCircle} & \textbf{Apoio personalizado:} Acesso direto para tirar dúvidas sobre estatística e Jamovi. \\[1mm]
\textcolor{red}{\faLock} & \textbf{Conteúdo exclusivo:} Vídeos, tutoriais e materiais especiais apenas para membros. \\[1mm]
\textcolor{red}{\faUsers} & \textbf{Proximidade com o autor:} Prioridade nas respostas. \\[1mm]
\textcolor{red}{\faThumbsUp} & \textbf{Reconhecimento:} Seu nome em destaque no canal do YouTube.
\end{tabular}

% Como se tornar um membro
\vspace{3mm}
\begin{center}
\textbf{Como se tornar um membro?}
\end{center}

\begin{center}
\begin{tcolorbox}[
  colback=red!5!white,
  colframe=red!60!black,
  arc=3mm,
  width=10cm,
  halign=center,
  valign=center,
  boxrule=1pt,
  sharp corners=all,
  rounded corners=northwest,
  rounded corners=northeast,
  rounded corners=southwest,
  rounded corners=southeast
]
\href{https://www.youtube.com/channel/UCzHAqwm3RSDM-YKdxd4XYTg/join}{\textcolor{red}{\faYoutube}\ \textbf{Tornar-se membro do Balaio Científico}}
\end{tcolorbox}
\end{center}

% Conteúdo exclusivo
\vspace{2mm}
\begin{center}
\textbf{Conteúdo exclusivo para membros}
\end{center}

Membros têm acesso à nossa biblioteca de vídeos exclusivos, incluindo tutoriais avançados, análise de casos reais e dicas especiais para otimizar seu trabalho com estatística e o Jamovi:

\vspace{2mm}
\begin{center}
\begin{tcolorbox}[
  colback=blue!5!white,
  colframe=blue!60!black,
  arc=3mm,
  width=10cm,
  halign=center,
  valign=center,
  boxrule=1pt,
  sharp corners=all,
  rounded corners=northwest,
  rounded corners=northeast,
  rounded corners=southwest,
  rounded corners=southeast
]
\href{https://www.youtube.com/playlist?list=UUMOzHAqwm3RSDM-YKdxd4XYTg}{\textcolor{blue}{\faPlayCircle}\ \textbf{Acessar conteúdo exclusivo para membros}}
\end{tcolorbox}
\end{center}

% Assinatura
\vspace{3mm}
\begin{flushright}
\textit{Com gratidão,}\\
\textbf{Erick Faria}\\
\textcolor{red}{\small\faYoutube}\ \small{ \href{https://www.youtube.com/@BalaioCient%C3%ADfico}{@BalaioCientifico}}
\end{flushright}

\newpage

%----------------------------------------------------------------------------------------
%	PÁGINA DE LICENÇA DO DOCUMENTO
%----------------------------------------------------------------------------------------

\begin{center}
\Huge\textbf{Licença do Documento}
\end{center}

Prezado(a) leitor(a),

Gostaria de ressaltar que este documento está licenciado sob a licença \href{https://creativecommons.org/licenses/by/4.0/deed.pt_BR}{Atribuição 4.0 Internacional (CC BY 4.0)}, o que significa que você tem a liberdade de usar, compartilhar e adaptar o material. 

No entanto, é importante fornecer as devidas atribuições ao autor original. Essas atribuições são uma maneira de reconhecer e valorizar o trabalho que foi dedicado para criar esta apostila.

% Direitos e termos da licença
\vfill
\small{\noindent \textbf{Você tem o direito de:}} \vspace{-3mm}\\
\noindent \rule{3.3cm}{0.5pt} \\
\textbf{Compartilhar } — copiar e redistribuir o material em qualquer suporte ou formato \\
\textbf{Adaptar} — remixar, transformar, e criar a partir do material para qualquer fim, mesmo que comercial. \\
\\
\small{\noindent \textbf{De acordo com os termos seguintes:}}\vspace{-3mm}\\
\noindent \rule{3.3cm}{0.5pt} \\
\textbf{Atribuição} — Você deve dar o crédito apropriado, prover um link para a licença e indicar se mudanças foram feitas. Você deve fazê-lo em qualquer circunstância razoável, mas de nenhuma maneira que sugira que o licenciante apoia você ou o seu uso.  \\
\textbf{Sem restrições adicionais} - Você não pode aplicar termos jurídicos ou medidas de caráter tecnológico que restrinjam legalmente outros de fazerem algo que a licença permita.  \\
\vspace{10mm} \\
\centerline{\href{https://creativecommons.org/licenses/by/4.0/deed.pt_BR}{Atribuição 4.0 Internacional (CC BY 4.0)}}
\vspace{5mm}\\
\centerline{\ccby}
