\chapter{Introdução ao Jamovi}

\section{O que é o Jamovi?}

O \textcite{Jamovi2022} é um software estatístico com interface gráfica. É um software relativamente novo, se comparado com os seus concorrentes como o SPSS, SAS e PSPP. Com o jamovi você consegue ter um ambiente de fácil aprendizado e fácil manuseio, pois é possível integrar a facilidade do ambiente gráfico, com o poder da linguagem R para a automação de trabalhos.

Além das análises estatísticas convencionais, tais como: estatística descritiva, tabelas cruzadas, boxplot, etc. É possível desenvolver modelos matemáticos. Assim, podemos definir o jamovi como uma solução completa para análises quantitativas e qualitativas com o uso da matemática e estatística. Um software estatístico completo e gratuito.

\subsection{Um software estatístico completo}

O jamovi é um dos softwares mais completos da atualidade. Sua interface gráfica e a integração com a linguagem de programação R, faz com que o programa seja capaz de realizar todas as análises, das mais simples até as mais importantes

Além de ser completo é um software open source e gratuito. Isso significa que você não precisa se preocupar em comprar licença e/ou fazer assinaturas. Além de ser gratuito, o fato de os códigos serem abertos, permite que todos e todas possam verificar diretamente no código fonte como o programa foi escrito, trazendo mais transparência e segurança para os(as) usuários(as)

\subsection{Estatística com interface gráfica}

Uma das principais barreiras para várias pessoas que desejam estudar estatística é a programação. Geralmente os(as) alunos(as) que estão começando a estudar estatística, estão começando também a ter o primeiro contato com a linguagem de programação

Uma das principais vantagens do Jamovi é sua interface gráfica. Com ela, o(a) aluno(a) que está começando nos estudos de estatística, pode se concentrar apenas no estudo teórico, deixando para outro momento o estudo da programação. Dessa forma, o jamovi é uma das melhores alternativas de software para estudos em estatística, pois permite que o(a) aluno(a) possa ver os resultados em tempo real, sem a necessidade de aprender os comandos de uma linguagem

\subsection{Software Modular}

O jamovi é um software modular. Isso significa que você pode instalar complementos, ou módulos de acordo com a sua necessidade. Diferentemente do SPSS, por exemplo, com o Jamovi você tem a liberdade de instalar somente os módulos que você precisa; algo que não é possível fazer no SPSS.

A liberdade de poder instalar os módulos, faz do Jamovi um software leve e personalizável. Além disso, a comunidade está ativamente desenvolvendo novos módulos, o que faz com que o Jamovi seja cada vez mais completo e flexível para atender diversos usuários

\subsection{Ambiente para aprender a linguagem R}

Apesar de ser um software com interface gráfica, o Jamovi tem integração com a linguagem de programação R. Isso significa que você tem a facilidade da interface gráfica, com o poder e a flexibilidade de uma das linguagens mais utilizadas no mundo da estatística.

Caso o Jamovi não tenha nativamente uma função que você precisa, você pode desenvolver utilizando a linguagem R. Essa integração faz com que seja possível realizar qualquer tipo de análise com o Jamovi. Além disso, você pode automatizar rotinas que são repetitivas e você precisa realizar várias vezes no seu trabalho e estudo.

\begin{tcolorbox}[colback=white,colframe=green,title= Dica de Conteúdo]
  Preparei um vídeo super especial de introdução ao jamovi, que vai te ajudar a dar os primeiros passos nessa incrível ferramenta estatística. Depois de ler a seção da apostila, não deixe de conferir essa videoaula, tenho certeza de que você vai adorar e aprender muito. Clique no link abaixo e aproveite!\\
  \faYoutube{} \href{https://www.youtube.com/watch?v=0zGH20Fa_JA&t=2s}{Introdução ao Jamovi}
\end{tcolorbox}

\section{Instalação do Jamovi}

Para fazer a instalação do software Jamovi, é muito simples e dependerá apenas do sistema operacional em que você está utilizando.

Se você fizer a instalação no site oficial você não precisa se preocupar com a sua segurança e está livre para instalar e fazer o uso do programa da forma que você precisar. Não existe nenhum tipo de restrição quanto ao uso ou locais em que você deve instalar.

Assim, você é livre para fazer uso educacional, comercial ou qualquer outro motivo pois não há nenhum tipo de restrição quanto ao uso ou a licença do programa.

\subsection{Download do Jamovi}

Para você fazer a utilização do jamovi, existem duas possibilidades: uma em cloud sendo que você não precisa realizar nenhum tipo de instalação e pode utilizar o programa diretamente da internet; em uma opção localmente em que você pode fazer o download no seu computador e instalar como um programa qualquer.

As duas soluções são muito boas, entretanto eu aconselho que você faça a instalação no seu computador para que você tenha total controle sobre os dados e não corra o risco de perder dados na solução de nuvem.

Eu recomendo que você faça a utilização do programa em nuvem somente em casos em que você não tem a possibilidade de instalar o jamovi em seu computador.

Nas duas sessões posteriores vou ensinar para vocês um pouco de como prosseguir para fazer a utilização das duas formas.

\subsection{Jamovi para desktop}

Primeiramente é importante que você acesse a página oficial do Jamovi para fazer o download. Você pode acessá-la na seguinte página: \url{https://www.jamovi.org}. É fortemente aconselhável que você faça o download somente na página oficial. Vale lembrar que o Jamovi é gratuito e não é necessário recorrer e nenhum programa paralelo, ou site diferente do oficial.

Para fazer a instalação em seu computador você deve selecionar a opção correspondente para instalação no desktop. No momento em que eu preparo esse material o site está organizado da seguinte forma.

Ao entrar na parte de download você verá uma página como na figura \ref{fig:download_jamovi}.

\begin{figure}[H]
  \centering
  \caption{Opção de Download do Jamovi para Desktop}
  \includegraphics[width=\textwidth]{imagens/cap_1/download_jamovi.png}
  \label{fig:download_jamovi}
\end{figure}

Selecione a opção desktop conforme indicado na imagem \ref{fig:download_jamovi} com uma seta. Ao clicar avance para o próximo tópico da apostila, pois vamos selecionar a versão do Jamovi que será instalada.

\subsection{Selecionando a versão do Jamovi}

Para fazer o download do Jamovi em seu computador, você precisa selecionar a versão correspondente ao seu sistema operacional. É importante que você faça a escolha correta, pois caso você cometa um equívoco na escola o programa não funcionará, ou não funcionará corretamente.

Geralmente quando você chegar na página de download, o próprio site já irá indicar a versão correta, correspondente ao seu sistema operacional. Basta você conferir e prosseguir para o download do programa.

Caso a sugestão do site esteja errada, basta rolar a página um pouco para baixo e selecionar a versão correta correspondente, assim como mostrado na imagem abaixo.

\begin{figure}[H]
  \centering
  \caption{Seleciona a versão do Jamovi}
  \includegraphics[width=0.5\textwidth]{imagens/cap_1/seleciona_versao_jamovi.png}
  \label{fig:download_jamovi}
\end{figure}

Quando você selecionar a versão que deseja utilizar, o download irá iniciar automaticamente.

\section{Instalando o Jamovi}

Após fazer o download do Jamovi, chegou o momento de você fazer a instalação em seu computador. Clique no arquivo que você acabou de baixar e siga as instuções para instalação.

O processo de instalação é simples e segue o mesmo fluxo de qualquer programa que você já está habituado(a) a fazer. Basta ir clicando em prosseguir, indicar um caminho de instalação, caso você queira selecionar uma pasta diferente dá pasta padrão e concluir a instalação.

\subsection{Jamovi em nuvem – cloud}

Na versão cloud você deve selecionar a opção correspondente como apontado na imagem \ref{fig:jamovi-cloud}.

\begin{figure}[H]
  \centering
  \caption{Versão cloud do Jamovi}
  \includegraphics[width=\textwidth]{imagens/cap_1/jamovi-cloud.png}
  \label{fig:jamovi-cloud}
\end{figure}

Ao clicar na versão cloud você irá se deparar com duas opções de utilização. Uma gratuita e outra paga.

Cabe você decidir se vale a pena pagar ou não para utilizar o Jamovi em cloud. A diferença da versão paga para a gratuita é apenas nos recursos computacionais e na disponibilidade do servidor onde o Jamovi é executado.

Entenda que executar um software em nuvem requer a utilização de infraestrutura e se um programa está sendo executado e não é em seu computador, algum lugar está fazendo para você.

Veja na imagem abaixo as opções que você pode escolher:

\begin{figure}[H]
  \centering
  \caption{Opções de utilização do Jamovi em cloud}
  \includegraphics[width=0.5\textwidth]{imagens/cap_1/opcoes_cloud.png}
  \label{fig:jamovi-cloud}
\end{figure}

Eu sou um grande entusiasta das soluções que são executadas em nuvem. Quase todos os meus trabalhos eu faço questão de utilizar algumas soluções em nuvem para realizar.

Caso eu fizesse o uso intensivo do jamovi em algum projeto meu, eu faria a assinatura sem problema. Entretanto se você é um estudante e não tem muitos recursos para fazer uma assinatura nesse momento, use a versão gratuita para realizar os seus estudos ou então ignore essa opção de utilização em nuvem e faça a utilização do programa instalando em seu computador normalmente.

Caso um dia você veja que faz sentido para você fazer assinatura para utilizar em algum projeto específico você vem faz a assinatura e faz a utilização normalmente. Assim, quando você for utilizar você já será um usuário avançado e poderá tirar todos os benefícios da solução em nuvem.

\section{Primeiros passos com o Jamovi}

Ao abrir o Jamovi pela primeira vez em seu computador, você verá uma imagem semelhante a esta mostrada abaixo. Essa é a interface gráfica do Jamovi e é onde você verá os dados e gráficos.

\begin{figure}[H]
  \centering
  \caption{Captura de Tela do Jamovi}
  \includegraphics[width=\textwidth]{imagens/cap_1/captura_tela_jamovi.png}
  \label{fig:captura_tela_jamovi}
\end{figure}

\begin{tcolorbox}[colback=white,colframe=green,title={\faPlayCircle \ Dica de Conteúdo}]
  Se você encontrou dificuldades nessa etapa, não se preocupe. Assista a videoaula a seguir em que eu ensino como dar os primeiros passos com o Jamovi. Nessa videoaula você será capaz de ver com maior clareza as etapas para iniciar no Jamovi.\\
  \faYoutube{} \href{https://www.youtube.com/watch?v=bV9hlHPLe5I&t=5s}{Primeiros Passos}
\end{tcolorbox}

\section{Instalando Módulos no Jamovi}

Uma das características mais poderosas do Jamovi é sua natureza modular. Diferentemente de softwares estatísticos tradicionais que vêm com todas as funcionalidades já instaladas (muitas vezes tornando o programa pesado e complexo), o Jamovi permite que você personalize sua instalação adicionando apenas os módulos que realmente necessita para seu trabalho.

\subsection{O que são módulos no Jamovi?}

Os módulos são conjuntos de funcionalidades adicionais que podem ser instalados no Jamovi para expandir suas capacidades. Podemos compará-los a plugins ou extensões que adicionam novos recursos ao programa base. Essa abordagem modular oferece várias vantagens:

\begin{itemize}
    \item \textbf{Desempenho otimizado}: Mantenha seu Jamovi leve instalando apenas o que você realmente utiliza
    \item \textbf{Personalização}: Adapte o software às suas necessidades específicas
    \item \textbf{Atualização constante}: Novos módulos são desenvolvidos regularmente pela comunidade
    \item \textbf{Especialização}: Acesso a funcionalidades avançadas para áreas específicas de pesquisa
\end{itemize}

\subsection{Acessando a biblioteca de módulos}

Para acessar e gerenciar os módulos no Jamovi:

\begin{enumerate}
    \item Clique no menu principal (três linhas horizontais) no canto superior direito da interface
    \item Selecione a opção \textbf{Módulos}
    \item Uma janela se abrirá com três abas principais:
    \begin{itemize}
        \item \textbf{Instalados}: Lista todos os módulos atualmente instalados
        \item \textbf{Biblioteca jamovi}: Mostra todos os módulos oficiais disponíveis para instalação
        \item \textbf{Módulo acessório}: Permite instalar módulos de fontes externas (arquivos .jmo)
    \end{itemize}
\end{enumerate}

% \begin{figure}[H]
%     \centering
%     \caption{Menu de módulos do Jamovi}
%     \includegraphics[width=\textwidth]{imagens/cap_1/modulos_jamovi_1.jpg}
%     \label{fig:modulos_jamovi_1}
% \end{figure}

\subsection{Módulos pré-instalados}

O Jamovi vem com alguns módulos básicos pré-instalados que fornecem as funcionalidades estatísticas essenciais:

\begin{itemize}
    \item \textbf{jmv}: O módulo principal com análises estatísticas básicas
    \item \textbf{Base R}: Integração básica com R
    \item \textbf{Descriptives}: Estatísticas descritivas e tabelas de contingência
    \item \textbf{T-Tests}: Testes t para amostras independentes e pareadas
    \item \textbf{ANOVA}: Análise de variância
    \item \textbf{Regression}: Análises de regressão linear
    \item \textbf{Frequencies}: Tabelas de frequência e análises categóricas
\end{itemize}

\subsection{Instalando novos módulos}

Para instalar um novo módulo no Jamovi:

\begin{enumerate}
    \item Na janela de módulos, clique na aba \textbf{Biblioteca jamovi}
    \item Navegue pela lista de módulos disponíveis ou role para baixo até encontrar o módulo desejado
    \item Clique no botão \textbf{Instalar} ao lado do módulo escolhido
    \item Aguarde a conclusão da instalação (uma barra de progresso será exibida)
    \item Após a instalação, o módulo estará imediatamente disponível para uso
\end{enumerate}

% \begin{figure}[H]
%     \centering
%     \caption{Biblioteca de módulos disponíveis}
%     \includegraphics[width=\textwidth]{imagens/cap_1/modulos_jamovi_2.jpg}
%     \label{fig:modulos_jamovi_2}
% \end{figure}

\subsection{Módulos populares e úteis}

Alguns módulos particularmente úteis que você pode considerar instalar:

\begin{itemize}
    \item \textbf{jpower}: Para cálculos de poder estatístico e tamanho amostral
    \item \textbf{Rj}: Editor R integrado que permite executar código R diretamente no Jamovi
    \item \textbf{scatr}: Gráficos de dispersão avançados e personalizáveis
    \item \textbf{flexplot}: Visualizações de dados flexíveis e intuitivas
    \item \textbf{jsq}: Análises estatísticas para pesquisas e questionários
    \item \textbf{jbr}: Análises bayesianas
    \item \textbf{jamm}: Modelos mistos avançados
    \item \textbf{DataCheck}: Ferramenta para verificar a qualidade dos dados
    \item \textbf{R datasets}: Acesso a conjuntos de dados clássicos de R para aprendizado
\end{itemize}

\subsection{Gerenciando módulos instalados}

Após instalar diversos módulos, é importante saber como gerenciá-los:

\begin{enumerate}
    \item Para visualizar os módulos instalados, clique na aba \textbf{Instalados}
    \item Para remover um módulo, selecione-o na lista e clique no botão \textbf{Remover}
    \item A remoção é instantânea e libera espaço e recursos do sistema
\end{enumerate}

\subsection{Instalando módulos de fontes externas}

Ocasionalmente, você pode querer instalar módulos que não estão disponíveis na biblioteca oficial do Jamovi:

\begin{enumerate}
    \item Obtenha o arquivo do módulo (.jmo) da fonte externa
    \item Na janela de módulos, clique na aba \textbf{Módulo acessório}
    \item Clique em \textbf{Procurar} e navegue até o arquivo .jmo em seu computador
    \item Selecione o arquivo e confirme a instalação
\end{enumerate}

Esta opção é particularmente útil para testar módulos em desenvolvimento ou para acessar módulos personalizados criados para necessidades específicas.

\subsection{Recomendações para uso eficiente de módulos}

Para otimizar sua experiência com os módulos do Jamovi:

\begin{itemize}
    \item \textbf{Seja seletivo}: Instale apenas os módulos que realmente precisa
    \item \textbf{Teste antes de confiar}: Ao instalar um novo módulo para análises importantes, teste-o com dados conhecidos
    \item \textbf{Verifique a documentação}: Muitos módulos possuem documentação específica que explica suas funcionalidades
    \item \textbf{Atualize regularmente}: Os módulos são frequentemente atualizados com novos recursos e correções
    \item \textbf{Desinstale o que não usa}: Remova módulos que não utiliza para manter o Jamovi rápido e eficiente
\end{itemize}

\begin{tcolorbox}[colback=white,colframe=green,title={\faPlayCircle \ Dica de Conteúdo}]
  Para ver na prática como instalar e gerenciar módulos no Jamovi, confira meu tutorial em vídeo. Nele, demonstro o processo passo a passo e apresento alguns dos módulos mais úteis que podem expandir significativamente as capacidades do seu Jamovi.\\
  \faYoutube{} \href{https://youtu.be/qBeRMNldzgs?si=TRYPqcPQeUFthNuo}{Como Instalar Módulos no Jamovi}
\end{tcolorbox}

% \subsection{Importando dados no Jamovi}

% Neste momento, é chegada a hora de realizar a leitura dos dados no Jamovi. Esse processo é conhecido por diversos termos, como importação, leitura ou carregamento de dados no software. Particularmente, opto por utilizar o termo importação de dados no Jamovi, embora seja válido utilizar a denominação de sua preferência.

% \begin{tcolorbox}[colback=white,colframe=orange,title= Importante]
%   É altamente recomendável você faça backup regularmente de seus arquivos e conteúdos no Jamovi. Fazer um backup é uma medida preventiva essencial para evitar a perda de dados importantes. Realizar backups periódicos garante que os dados sejam protegidos contra possíveis falhas técnicas, erros humanos ou até mesmo perda acidental de arquivos.
% \end{tcolorbox}

% Nessa seção eu vou utilizar os dados de passageiros do titanic como exemplo para ensinar a importação de dados do jamovi. Caso você tenha interesse de utilizar o mesmo conjunto de dados que eu, basta fazer o download do arquivo csv no \href{https://github.com/balaio-cientifico/dataset}{repositório oficial do Balaio Científico}.

% Com o Jamovi aberto, acesse o menu principal do software clicando no ícone localizado no canto superior esquerdo da tela. Ao realizar essa ação, uma nova janela será exibida. Observe a imagem abaixo para identificar o ícone a ser selecionado, indicado pela seta.

% \begin{figure}[H]
%   \centering
%   \caption{Captura de Tela do Jamovi}
%   \includegraphics[width=\textwidth]{imagens/cap_1/importa_dados_1.png}
%   \label{fig:importa_dados_1}
% \end{figure}

% Após clicar em “Abrir”, selecione a opção “Este PC”. No exemplo apresentado, alguns arquivos que já foram abertos anteriormente são exibidos. Entretanto, se este for o primeiro uso do Jamovi, é necessário clicar em “Procurar” e navegar até o local em que o arquivo de interesse está salvo para, então, importá-lo no software. Para localizar o arquivo, basta navegar pelas pastas até encontrar o diretório em que o arquivo está armazenado.

% \begin{figure}[H]
%   \centering
%   \caption{Captura de Tela do Jamovi}
%   \includegraphics[width=\textwidth]{imagens/cap_1/importa_dados_2.png}
%   \label{fig:importa_dados_2}
% \end{figure}

% Ao clicar no arquivo que contém o conjunto de dados, o Jamovi realizará automaticamente a configuração e importação dos dados. Se você tiver importado os mesmos dados do exemplo apresentado, uma janela semelhante à imagem abaixo será exibida.

% \begin{figure}[H]
%   \centering
%   \caption{Captura de Tela do Jamovi}
%   \includegraphics[width=\textwidth]{imagens/cap_1/jamovi_arquivo_importado.png}
%   \label{fig:jamovi_arquivo_importado}
% \end{figure}

% Após a importação dos dados, você estará pronto(a) para começar a realizar análises estatísticas e utilizar todas as funcionalidades disponíveis no Jamovi. É importante praticar a importação de diferentes conjuntos de dados para se familiarizar com o processo e estar apto(a) a realizá-lo com facilidade.

% \begin{tcolorbox}[colback=white,colframe=green,title= Dica de Conteúdo]
%   Se você teve dificuldades para fazer a importação dos dados não se preocupe. Essa video aula vai te mostrarcom todos os detalhes como você pode fazer a importação dos dados do Jamovi. Destaco que essa é uma das etapas mais importantes, logo tenha certeza de que entendeu corretamente para que seu aprendizado não fique prejudicado.\\
%   \faYoutube{} \href{https://www.youtube.com/watch?v=NIpt0wIq5pc&t=1s}{Importar Dados no Jamovi}
% \end{tcolorbox}