\documentclass[12pt, oneside]{book}

% Pacotes necessários
\usepackage[portuguese]{babel}
\usepackage[utf8]{inputenc}
\usepackage[scale=1.5]{ccicons}
\usepackage{graphicx}
\usepackage{titlesec}
\usepackage{setspace}
\usepackage{geometry}
\usepackage{indentfirst}
\usepackage{hyperref}
\usepackage{float}
\usepackage{tikz}
\usepackage{kpfonts}
\usepackage{fontawesome5}
\usepackage{fancyhdr}
\usepackage{subfigure}
\usepackage{tcolorbox}
\usepackage{xcolor}
\usepackage{pgfplots}
\usepackage{pgfmath}

\usepackage[style=abnt]{biblatex}

\addbibresource{bibliografia.bib}

\usetikzlibrary{intersections}

% Define o estilo da página
\pagestyle{fancy}
\fancyhf{}
\fancyfoot[L]{Erick Faria}
\fancyfoot[C]{\thepage}
\fancyfoot[R]{\href{https://www.balaiocientifico.com/?utm_source=apostila_jamovi&utm_medium=pdf}{Balaio Científico}}
\fancyhead[R]{\nouppercase{\leftmark}}
\renewcommand{\headrulewidth}{0.4pt}
\renewcommand{\footrulewidth}{0.4pt}
% Redefine o comando \chaptermark para remover o prefixo "Chapter X."
\renewcommand{\chaptermark}[1]{\markboth{#1}{}}


\usefont{T1}{jkpss}{m}{n}

\geometry{
  left=3cm,
  right=2cm,
  top=3cm,
  bottom=2cm,
  bindingoffset=0cm
}

\onehalfspacing

% Define o caminho das imagens
\graphicspath{{imagens/}}

% Define o estilo de início do capítulo
\titleformat{\chapter}[display]
  {\normalfont\huge\bfseries\filcenter}
  {\begin{tikzpicture}[remember picture,overlay]
    \node[anchor=north west,inner sep=0pt] at (current page.north west)
      {\includegraphics[width=\paperwidth,height=0.2\paperheight]{imagens/fundo_cap.pdf}};
    \fill[white,opacity=0] (current page.north west) rectangle (current page.south east);
    \draw[anchor=north west] (0.2\paperwidth,0.2\paperheight) node [font=\Huge\bfseries] {\chaptertitlename\ \thechapter};
  \end{tikzpicture}}
  {0pt}
  {\Huge}

% Define os metadados do documento
\hypersetup{
colorlinks=true,
linkcolor=blue,
filecolor=blue,
citecolor=blue,      
urlcolor=blue,
pdftitle={Apostila de Jamovi},
pdfauthor={Erick Faria},
pdfsubject={Apostila de Jamovi},
pdfkeywords={Jamovi, Estatística, Análise de Dados}
}

\begin{document}
\let\cleardoublepage\clearpage

% Inclui a capa
\begin{titlepage}
	\centering % Center everything on the title page
	\scshape % Use small caps for all text on the title page
	\vspace*{1.5\baselineskip} % White space at the top of the page
% ===================
%	Title Section 	
% ===================

	\rule{13cm}{1.6pt}\vspace*{-\baselineskip}\vspace*{2pt} % Thick horizontal rule
	\rule{13cm}{0.4pt} % Thin horizontal rule
	
		\vspace{0.75\baselineskip} % Whitespace above the title
% ========== Title ===============	
	{	\Huge Introdução ao\\ 
			\vspace{4mm}
		JAMOVI \\	}
% ======================================
		\vspace{0.75\baselineskip} % Whitespace below the title
	\rule{13cm}{0.4pt}\vspace*{-\baselineskip}\vspace{3.2pt} % Thin horizontal rule
	\rule{13cm}{1.6pt} % Thick horizontal rule
	
		\vspace{1.75\baselineskip} % Whitespace after the title block
% =================
%	Information	
% =================
	{\large Produzido por: \href{https://www.balaiocientifico.com/author/erickfaria/?utm_source=apostila_jamovi&utm_medium=pdf}{Erick Faria} \\
		\vspace*{1.2\baselineskip}}
	% \href{mailto:erickfaria@balaiocientifico.com}{erickfaria@balaiocientifico.com}} \\
	\vfill
	
\href{https://www.balaiocientifico.com/?utm_source=apostila_jamovi&utm_medium=pdf}{\texttt{www.balaiocientifico.com}}\\
\vspace{0.5cm}
\ccby
\end{titlepage}

% Define a numeração das páginas em romanos
\pagenumbering{roman}
\setcounter{page}{1}

% Inclui a página de atribuição
%----------------------------------------------------------------------------------------
%	PÁGINA DE APRESENTAÇÃO AO LEITOR
%----------------------------------------------------------------------------------------

\begin{flushleft}
\Huge\textbf{Prezado(a) leitor(a),}
\end{flushleft}

Fico feliz e agradeço por você ter escolhido ler a Apostila de Introdução ao Jamovi. Espero que este recurso seja útil para você em sua jornada de aprendizado e aplicação de análises estatísticas com o Jamovi.

É importante mencionar também que a apostila está em constante atualização. À medida que novas versões forem lançadas, novos recursos, exemplos e aprimoramentos serão adicionados. Recomento que você visite regularmente o site do \href{https://www.balaiocientifico.com/jamovi/apostila-de-jamovi/?utm_source=apostila_jamovi&utm_medium=pdf} {Balaio Científico} para obter a versão mais atualizada da apostila. Dessa forma, você poderá se beneficiar de novos conteúdos e melhorias à medida que são disponibilizados.

Além disso, incentivo você a compartilhar e usar este material com outras pessoas interessadas em aprender sobre análise estatística com o Jamovi. Acredito no poder do conhecimento compartilhado e na colaboração para promover um aprendizado mais amplo e significativo.

Se você deseja colaborar com esse projeto, você pode acessar o repositório dessa apostila no \href{https://github.com/balaio-cientifico/apostila_jamovi}{Github}. Se quiser entrar em contato comigo você pode me enviar um e-mail, ou entrar em contato comigo por uma das redes sociais listadas abaixo.

Mais uma vez, agradeço sua leitura e interesse pela Apostila do Jamovi. Espero que ela seja útil para você em seus estudos e projetos. Se você tiver alguma dúvida, sugestão ou feedback, não hesite em entrar em contato comigo. Estou sempre buscando melhorar e fornecer recursos de alta qualidade para a comunidade.

Aproveite o material e bons estudos!

\vfill

% Seção de redes sociais
\noindent \textbf{Me Siga nas redes:}

\noindent \faTwitter{} \href{https://twitter.com/balaioci}{balaioci} \\
\faFacebook{} \href{https://www.facebook.com/balaiocientifico/}{balaiocientifico} \\
\faYoutube{} \href{https://www.youtube.com/@BalaioCientifico}{@BalaioCientifico} \\
\faInstagram{} \href{https://www.instagram.com/balaiocientifico/}{balaiocientifico} \\
\faEnvelope{} \href{mailto:contato@balaiocientifico.com}{erickfaria@balaiocientifico.com}\\

% Seção de livros
\noindent \textbf{Conheça meus livros:}

\noindent \faAmazon{} \href{https://www.amazon.com.br/Erick-Faria/e/B09SZPGBZL/}{Amazon} \\

\vfill
\centerline{Atualizado em: {\noindent \today}}

\newpage

%----------------------------------------------------------------------------------------
%	PÁGINA DE APOIO AO BALAIO CIENTÍFICO
%----------------------------------------------------------------------------------------

\begin{center}
\Large\textbf{Apoie o Balaio Científico}
\end{center}

\vspace{2mm}

Caro(a) leitor(a),

No Balaio Científico, acreditamos que o conhecimento deve ser democrático. Por isso, a maior parte do nosso material — aulas, tutoriais e ferramentas — é disponibilizada gratuitamente para todos(as). Produzir material educativo de qualidade, no entanto, demanda tempo e recursos. É por isso que criei um programa de membros, onde você pode apoiar este trabalho e, em troca, receber benefícios exclusivos.

% Por que se tornar um membro
\vspace{1mm}
\begin{center}
\textbf{Por que se tornar um membro?}
\end{center}

\vspace{1mm}
\noindent\begin{tabular}{p{1cm}p{13.5cm}}
\textcolor{red}{\faQuestionCircle} & \textbf{Apoio personalizado:} Acesso direto para tirar dúvidas sobre estatística e Jamovi. \\[1mm]
\textcolor{red}{\faLock} & \textbf{Conteúdo exclusivo:} Vídeos, tutoriais e materiais especiais apenas para membros. \\[1mm]
\textcolor{red}{\faUsers} & \textbf{Proximidade com o autor:} Prioridade nas respostas. \\[1mm]
\textcolor{red}{\faThumbsUp} & \textbf{Reconhecimento:} Seu nome em destaque no canal do YouTube.
\end{tabular}

% Como se tornar um membro
\vspace{3mm}
\begin{center}
\textbf{Como se tornar um membro?}
\end{center}

\begin{center}
\begin{tcolorbox}[
  colback=red!5!white,
  colframe=red!60!black,
  arc=6mm,
  width=10cm,
  halign=center,
  valign=center
]
\href{https://www.youtube.com/channel/UCzHAqwm3RSDM-YKdxd4XYTg/join}{\textcolor{red}{\faYoutube}\ \textbf{Tornar-se membro do Balaio Científico}}
\end{tcolorbox}
\end{center}

% Conteúdo exclusivo
\vspace{2mm}
\begin{center}
\textbf{Conteúdo exclusivo para membros}
\end{center}

Membros têm acesso à nossa biblioteca de vídeos exclusivos, incluindo tutoriais avançados, análise de casos reais e dicas especiais para otimizar seu trabalho com estatística e o Jamovi:

\vspace{2mm}
\begin{center}
\begin{tcolorbox}[
  colback=blue!5!white,
  colframe=blue!60!black,
  arc=6mm,
  width=10cm,
  halign=center,
  valign=center
]
\href{https://www.youtube.com/playlist?list=UUMOzHAqwm3RSDM-YKdxd4XYTg}{\textcolor{blue}{\faPlayCircle}\ \textbf{Acessar conteúdo exclusivo para membros}}
\end{tcolorbox}
\end{center}

\vfill

\begin{center}
\emph{Seu apoio é fundamental para que eu possa continuar produzindo e atualizando conteúdos educativos de qualidade. Independentemente de sua escolha, agradeço por utilizar esta apostila!}
\end{center}

% Assinatura
\vspace{3mm}
\begin{flushright}
\textit{Com gratidão,}\\
\textbf{Erick Faria}\\
\textcolor{red}{\small\faYoutube}\ \small{ \href{https://www.youtube.com/@BalaioCient%C3%ADfico}{@BalaioCientifico}}
\end{flushright}

\newpage

%----------------------------------------------------------------------------------------
%	PÁGINA DE LICENÇA DO DOCUMENTO
%----------------------------------------------------------------------------------------

\begin{center}
\Huge\textbf{Licença do Documento}
\end{center}

Prezado(a) leitor(a),

Gostaria de ressaltar que este documento está licenciado sob a licença \href{https://creativecommons.org/licenses/by/4.0/deed.pt_BR}{Atribuição 4.0 Internacional (CC BY 4.0)}, o que significa que você tem a liberdade de usar, compartilhar e adaptar o material. 

No entanto, é importante fornecer as devidas atribuições ao autor original. Essas atribuições são uma maneira de reconhecer e valorizar o trabalho que foi dedicado para criar esta apostila.

% Direitos e termos da licença
\vfill
\small{\noindent \textbf{Você tem o direito de:}} \vspace{-3mm}\\
\noindent \rule{3.3cm}{0.5pt} \\
\textbf{Compartilhar } — copiar e redistribuir o material em qualquer suporte ou formato \\
\textbf{Adaptar} — remixar, transformar, e criar a partir do material para qualquer fim, mesmo que comercial. \\
\\
\small{\noindent \textbf{De acordo com os termos seguintes:}}\vspace{-3mm}\\
\noindent \rule{3.3cm}{0.5pt} \\
\textbf{Atribuição} — Você deve dar o crédito apropriado, prover um link para a licença e indicar se mudanças foram feitas. Você deve fazê-lo em qualquer circunstância razoável, mas de nenhuma maneira que sugira que o licenciante apoia você ou o seu uso.  \\
\textbf{Sem restrições adicionais} - Você não pode aplicar termos jurídicos ou medidas de caráter tecnológico que restrinjam legalmente outros de fazerem algo que a licença permita.  \\
\vspace{10mm} \\
\centerline{\href{https://creativecommons.org/licenses/by/4.0/deed.pt_BR}{Atribuição 4.0 Internacional (CC BY 4.0)}}
\vspace{5mm}\\
\centerline{\ccby}


% Define o nome do sumário
\renewcommand{\contentsname}{Sumário}

% Inclui o sumário
\tableofcontents

% Define a numeração das páginas em arábicos
\pagenumbering{arabic}
\setcounter{page}{1}

% Inclui o primeiro capítulo
\chapter{Introdução ao Jamovi}

\section{O que é o Jamovi?}

O \textcite{Jamovi2022} é um software estatístico com interface gráfica. É um software relativamente novo, se comparado com os seus concorrentes como o SPSS, SAS e PSPP. Com o jamovi você consegue ter um ambiente de fácil aprendizado e fácil manuseio, pois é possível integrar a facilidade do ambiente gráfico, com o poder da linguagem R para a automação de trabalhos.

Além das análises estatísticas convencionais, tais como: estatística descritiva, tabelas cruzadas, boxplot, etc. É possível desenvolver modelos matemáticos. Assim, podemos definir o jamovi como uma solução completa para análises quantitativas e qualitativas com o uso da matemática e estatística. Um software estatístico completo e gratuito.

\subsection{Um software estatístico completo}

O jamovi é um dos softwares mais completos da atualidade. Sua interface gráfica e a integração com a linguagem de programação R, faz com que o programa seja capaz de realizar todas as análises, das mais simples até as mais importantes

Além de ser completo é um software open source e gratuito. Isso significa que você não precisa se preocupar em comprar licença e/ou fazer assinaturas. Além de ser gratuito, o fato de os códigos serem abertos, permite que todos e todas possam verificar diretamente no código fonte como o programa foi escrito, trazendo mais transparência e segurança para os(as) usuários(as)

\subsection{Estatística com interface gráfica}

Uma das principais barreiras para várias pessoas que desejam estudar estatística é a programação. Geralmente os(as) alunos(as) que estão começando a estudar estatística, estão começando também a ter o primeiro contato com a linguagem de programação

Uma das principais vantagens do Jamovi é sua interface gráfica. Com ela, o(a) aluno(a) que está começando nos estudos de estatística, pode se concentrar apenas no estudo teórico, deixando para outro momento o estudo da programação. Dessa forma, o jamovi é uma das melhores alternativas de software para estudos em estatística, pois permite que o(a) aluno(a) possa ver os resultados em tempo real, sem a necessidade de aprender os comandos de uma linguagem

\subsection{Software Modular}

O jamovi é um software modular. Isso significa que você pode instalar complementos, ou módulos de acordo com a sua necessidade. Diferentemente do SPSS, por exemplo, com o Jamovi você tem a liberdade de instalar somente os módulos que você precisa; algo que não é possível fazer no SPSS.

A liberdade de poder instalar os módulos, faz do Jamovi um software leve e personalizável. Além disso, a comunidade está ativamente desenvolvendo novos módulos, o que faz com que o Jamovi seja cada vez mais completo e flexível para atender diversos usuários

\subsection{Ambiente para aprender a linguagem R}

Apesar de ser um software com interface gráfica, o Jamovi tem integração com a linguagem de programação R. Isso significa que você tem a facilidade da interface gráfica, com o poder e a flexibilidade de uma das linguagens mais utilizadas no mundo da estatística.

Caso o Jamovi não tenha nativamente uma função que você precisa, você pode desenvolver utilizando a linguagem R. Essa integração faz com que seja possível realizar qualquer tipo de análise com o Jamovi. Além disso, você pode automatizar rotinas que são repetitivas e você precisa realizar várias vezes no seu trabalho e estudo.

\begin{tcolorbox}[colback=white,colframe=green,title= Dica de Conteúdo]
  Preparei um vídeo super especial de introdução ao jamovi, que vai te ajudar a dar os primeiros passos nessa incrível ferramenta estatística. Depois de ler a seção da apostila, não deixe de conferir essa videoaula, tenho certeza de que você vai adorar e aprender muito. Clique no link abaixo e aproveite!\\
  \faYoutube{} \href{https://www.youtube.com/watch?v=0zGH20Fa_JA&t=2s}{Introdução ao Jamovi}
\end{tcolorbox}

\section{Instalação do Jamovi}

Para fazer a instalação do software Jamovi, é muito simples e dependerá apenas do sistema operacional em que você está utilizando.

Se você fizer a instalação no site oficial você não precisa se preocupar com a sua segurança e está livre para instalar e fazer o uso do programa da forma que você precisar. Não existe nenhum tipo de restrição quanto ao uso ou locais em que você deve instalar.

Assim, você é livre para fazer uso educacional, comercial ou qualquer outro motivo pois não há nenhum tipo de restrição quanto ao uso ou a licença do programa.

\subsection{Download do Jamovi}

Para você fazer a utilização do jamovi, existem duas possibilidades: uma em cloud sendo que você não precisa realizar nenhum tipo de instalação e pode utilizar o programa diretamente da internet; em uma opção localmente em que você pode fazer o download no seu computador e instalar como um programa qualquer.

As duas soluções são muito boas, entretanto eu aconselho que você faça a instalação no seu computador para que você tenha total controle sobre os dados e não corra o risco de perder dados na solução de nuvem.

Eu recomendo que você faça a utilização do programa em nuvem somente em casos em que você não tem a possibilidade de instalar o jamovi em seu computador.

Nas duas sessões posteriores vou ensinar para vocês um pouco de como prosseguir para fazer a utilização das duas formas.

\subsection{Jamovi para desktop}

Primeiramente é importante que você acesse a página oficial do Jamovi para fazer o download. Você pode acessá-la na seguinte página: \url{https://www.jamovi.org}. É fortemente aconselhável que você faça o download somente na página oficial. Vale lembrar que o Jamovi é gratuito e não é necessário recorrer e nenhum programa paralelo, ou site diferente do oficial.

Para fazer a instalação em seu computador você deve selecionar a opção correspondente para instalação no desktop. No momento em que eu preparo esse material o site está organizado da seguinte forma.

Ao entrar na parte de download você verá uma página como na figura \ref{fig:download_jamovi}.

\begin{figure}[H]
  \centering
  \caption{Opção de Download do Jamovi para Desktop}
  \includegraphics[width=\textwidth]{imagens/cap_1/download_jamovi.png}
  \label{fig:download_jamovi}
\end{figure}

Selecione a opção desktop conforme indicado na imagem \ref{fig:download_jamovi} com uma seta. Ao clicar avance para o próximo tópico da apostila, pois vamos selecionar a versão do Jamovi que será instalada.

\subsection{Selecionando a versão do Jamovi}

Para fazer o download do Jamovi em seu computador, você precisa selecionar a versão correspondente ao seu sistema operacional. É importante que você faça a escolha correta, pois caso você cometa um equívoco na escola o programa não funcionará, ou não funcionará corretamente.

Geralmente quando você chegar na página de download, o próprio site já irá indicar a versão correta, correspondente ao seu sistema operacional. Basta você conferir e prosseguir para o download do programa.

Caso a sugestão do site esteja errada, basta rolar a página um pouco para baixo e selecionar a versão correta correspondente, assim como mostrado na imagem abaixo.

\begin{figure}[H]
  \centering
  \caption{Seleciona a versão do Jamovi}
  \includegraphics[width=0.5\textwidth]{imagens/cap_1/seleciona_versao_jamovi.png}
  \label{fig:download_jamovi}
\end{figure}

Quando você selecionar a versão que deseja utilizar, o download irá iniciar automaticamente.

\section{Instalando o Jamovi}

Após fazer o download do Jamovi, chegou o momento de você fazer a instalação em seu computador. Clique no arquivo que você acabou de baixar e siga as instuções para instalação.

O processo de instalação é simples e segue o mesmo fluxo de qualquer programa que você já está habituado(a) a fazer. Basta ir clicando em prosseguir, indicar um caminho de instalação, caso você queira selecionar uma pasta diferente dá pasta padrão e concluir a instalação.

\subsection{Jamovi em nuvem – cloud}

Na versão cloud você deve selecionar a opção correspondente como apontado na imagem \ref{fig:jamovi-cloud}.

\begin{figure}[H]
  \centering
  \caption{Versão cloud do Jamovi}
  \includegraphics[width=\textwidth]{imagens/cap_1/jamovi-cloud.png}
  \label{fig:jamovi-cloud}
\end{figure}

Ao clicar na versão cloud você irá se deparar com duas opções de utilização. Uma gratuita e outra paga.

Cabe você decidir se vale a pena pagar ou não para utilizar o Jamovi em cloud. A diferença da versão paga para a gratuita é apenas nos recursos computacionais e na disponibilidade do servidor onde o Jamovi é executado.

Entenda que executar um software em nuvem requer a utilização de infraestrutura e se um programa está sendo executado e não é em seu computador, algum lugar está fazendo para você.

Veja na imagem abaixo as opções que você pode escolher:

\begin{figure}[H]
  \centering
  \caption{Opções de utilização do Jamovi em cloud}
  \includegraphics[width=0.5\textwidth]{imagens/cap_1/opcoes_cloud.png}
  \label{fig:jamovi-cloud}
\end{figure}

Eu sou um grande entusiasta das soluções que são executadas em nuvem. Quase todos os meus trabalhos eu faço questão de utilizar algumas soluções em nuvem para realizar.

Caso eu fizesse o uso intensivo do jamovi em algum projeto meu, eu faria a assinatura sem problema. Entretanto se você é um estudante e não tem muitos recursos para fazer uma assinatura nesse momento, use a versão gratuita para realizar os seus estudos ou então ignore essa opção de utilização em nuvem e faça a utilização do programa instalando em seu computador normalmente.

Caso um dia você veja que faz sentido para você fazer assinatura para utilizar em algum projeto específico você vem faz a assinatura e faz a utilização normalmente. Assim, quando você for utilizar você já será um usuário avançado e poderá tirar todos os benefícios da solução em nuvem.

\section{Primeiros passos com o Jamovi}

Ao abrir o Jamovi pela primeira vez em seu computador, você verá uma imagem semelhante a esta mostrada abaixo. Essa é a interface gráfica do Jamovi e é onde você verá os dados e gráficos.

\begin{figure}[H]
  \centering
  \caption{Captura de Tela do Jamovi}
  \includegraphics[width=\textwidth]{imagens/cap_1/captura_tela_jamovi.png}
  \label{fig:captura_tela_jamovi}
\end{figure}

\begin{tcolorbox}[colback=white,colframe=green,title={\faPlayCircle \ Dica de Conteúdo}]
  Se você encontrou dificuldades nessa etapa, não se preocupe. Assista a videoaula a seguir em que eu ensino como dar os primeiros passos com o Jamovi. Nessa videoaula você será capaz de ver com maior clareza as etapas para iniciar no Jamovi.\\
  \faYoutube{} \href{https://www.youtube.com/watch?v=bV9hlHPLe5I&t=5s}{Primeiros Passos}
\end{tcolorbox}

\section{Instalando Módulos no Jamovi}

Uma das características mais poderosas do Jamovi é sua natureza modular. Diferentemente de softwares estatísticos tradicionais que vêm com todas as funcionalidades já instaladas (muitas vezes tornando o programa pesado e complexo), o Jamovi permite que você personalize sua instalação adicionando apenas os módulos que realmente necessita para seu trabalho.

\subsection{O que são módulos no Jamovi?}

Os módulos são conjuntos de funcionalidades adicionais que podem ser instalados no Jamovi para expandir suas capacidades. Podemos compará-los a plugins ou extensões que adicionam novos recursos ao programa base. Essa abordagem modular oferece várias vantagens:

\begin{itemize}
    \item \textbf{Desempenho otimizado}: Mantenha seu Jamovi leve instalando apenas o que você realmente utiliza
    \item \textbf{Personalização}: Adapte o software às suas necessidades específicas
    \item \textbf{Atualização constante}: Novos módulos são desenvolvidos regularmente pela comunidade
    \item \textbf{Especialização}: Acesso a funcionalidades avançadas para áreas específicas de pesquisa
\end{itemize}

\subsection{Acessando a biblioteca de módulos}

Para acessar e gerenciar os módulos no Jamovi:

\begin{enumerate}
    \item Clique no menu principal (três linhas horizontais) no canto superior direito da interface
    \item Selecione a opção \textbf{Módulos}
    \item Uma janela se abrirá com três abas principais:
    \begin{itemize}
        \item \textbf{Instalados}: Lista todos os módulos atualmente instalados
        \item \textbf{Biblioteca jamovi}: Mostra todos os módulos oficiais disponíveis para instalação
        \item \textbf{Módulo acessório}: Permite instalar módulos de fontes externas (arquivos .jmo)
    \end{itemize}
\end{enumerate}

% \begin{figure}[H]
%     \centering
%     \caption{Menu de módulos do Jamovi}
%     \includegraphics[width=\textwidth]{imagens/cap_1/modulos_jamovi_1.jpg}
%     \label{fig:modulos_jamovi_1}
% \end{figure}

\subsection{Módulos pré-instalados}

O Jamovi vem com alguns módulos básicos pré-instalados que fornecem as funcionalidades estatísticas essenciais:

\begin{itemize}
    \item \textbf{jmv}: O módulo principal com análises estatísticas básicas
    \item \textbf{Base R}: Integração básica com R
    \item \textbf{Descriptives}: Estatísticas descritivas e tabelas de contingência
    \item \textbf{T-Tests}: Testes t para amostras independentes e pareadas
    \item \textbf{ANOVA}: Análise de variância
    \item \textbf{Regression}: Análises de regressão linear
    \item \textbf{Frequencies}: Tabelas de frequência e análises categóricas
\end{itemize}

\subsection{Instalando novos módulos}

Para instalar um novo módulo no Jamovi:

\begin{enumerate}
    \item Na janela de módulos, clique na aba \textbf{Biblioteca jamovi}
    \item Navegue pela lista de módulos disponíveis ou role para baixo até encontrar o módulo desejado
    \item Clique no botão \textbf{Instalar} ao lado do módulo escolhido
    \item Aguarde a conclusão da instalação (uma barra de progresso será exibida)
    \item Após a instalação, o módulo estará imediatamente disponível para uso
\end{enumerate}

% \begin{figure}[H]
%     \centering
%     \caption{Biblioteca de módulos disponíveis}
%     \includegraphics[width=\textwidth]{imagens/cap_1/modulos_jamovi_2.jpg}
%     \label{fig:modulos_jamovi_2}
% \end{figure}

\subsection{Módulos populares e úteis}

Alguns módulos particularmente úteis que você pode considerar instalar:

\begin{itemize}
    \item \textbf{jpower}: Para cálculos de poder estatístico e tamanho amostral
    \item \textbf{Rj}: Editor R integrado que permite executar código R diretamente no Jamovi
    \item \textbf{scatr}: Gráficos de dispersão avançados e personalizáveis
    \item \textbf{flexplot}: Visualizações de dados flexíveis e intuitivas
    \item \textbf{jsq}: Análises estatísticas para pesquisas e questionários
    \item \textbf{jbr}: Análises bayesianas
    \item \textbf{jamm}: Modelos mistos avançados
    \item \textbf{DataCheck}: Ferramenta para verificar a qualidade dos dados
    \item \textbf{R datasets}: Acesso a conjuntos de dados clássicos de R para aprendizado
\end{itemize}

\subsection{Gerenciando módulos instalados}

Após instalar diversos módulos, é importante saber como gerenciá-los:

\begin{enumerate}
    \item Para visualizar os módulos instalados, clique na aba \textbf{Instalados}
    \item Para remover um módulo, selecione-o na lista e clique no botão \textbf{Remover}
    \item A remoção é instantânea e libera espaço e recursos do sistema
\end{enumerate}

\subsection{Instalando módulos de fontes externas}

Ocasionalmente, você pode querer instalar módulos que não estão disponíveis na biblioteca oficial do Jamovi:

\begin{enumerate}
    \item Obtenha o arquivo do módulo (.jmo) da fonte externa
    \item Na janela de módulos, clique na aba \textbf{Módulo acessório}
    \item Clique em \textbf{Procurar} e navegue até o arquivo .jmo em seu computador
    \item Selecione o arquivo e confirme a instalação
\end{enumerate}

Esta opção é particularmente útil para testar módulos em desenvolvimento ou para acessar módulos personalizados criados para necessidades específicas.

\subsection{Recomendações para uso eficiente de módulos}

Para otimizar sua experiência com os módulos do Jamovi:

\begin{itemize}
    \item \textbf{Seja seletivo}: Instale apenas os módulos que realmente precisa
    \item \textbf{Teste antes de confiar}: Ao instalar um novo módulo para análises importantes, teste-o com dados conhecidos
    \item \textbf{Verifique a documentação}: Muitos módulos possuem documentação específica que explica suas funcionalidades
    \item \textbf{Atualize regularmente}: Os módulos são frequentemente atualizados com novos recursos e correções
    \item \textbf{Desinstale o que não usa}: Remova módulos que não utiliza para manter o Jamovi rápido e eficiente
\end{itemize}

\begin{tcolorbox}[colback=white,colframe=green,title={\faPlayCircle \ Dica de Conteúdo}]
  Para ver na prática como instalar e gerenciar módulos no Jamovi, confira meu tutorial em vídeo. Nele, demonstro o processo passo a passo e apresento alguns dos módulos mais úteis que podem expandir significativamente as capacidades do seu Jamovi.\\
  \faYoutube{} \href{https://youtu.be/qBeRMNldzgs?si=TRYPqcPQeUFthNuo}{Como Instalar Módulos no Jamovi}
\end{tcolorbox}

% \subsection{Importando dados no Jamovi}

% Neste momento, é chegada a hora de realizar a leitura dos dados no Jamovi. Esse processo é conhecido por diversos termos, como importação, leitura ou carregamento de dados no software. Particularmente, opto por utilizar o termo importação de dados no Jamovi, embora seja válido utilizar a denominação de sua preferência.

% \begin{tcolorbox}[colback=white,colframe=orange,title= Importante]
%   É altamente recomendável você faça backup regularmente de seus arquivos e conteúdos no Jamovi. Fazer um backup é uma medida preventiva essencial para evitar a perda de dados importantes. Realizar backups periódicos garante que os dados sejam protegidos contra possíveis falhas técnicas, erros humanos ou até mesmo perda acidental de arquivos.
% \end{tcolorbox}

% Nessa seção eu vou utilizar os dados de passageiros do titanic como exemplo para ensinar a importação de dados do jamovi. Caso você tenha interesse de utilizar o mesmo conjunto de dados que eu, basta fazer o download do arquivo csv no \href{https://github.com/balaio-cientifico/dataset}{repositório oficial do Balaio Científico}.

% Com o Jamovi aberto, acesse o menu principal do software clicando no ícone localizado no canto superior esquerdo da tela. Ao realizar essa ação, uma nova janela será exibida. Observe a imagem abaixo para identificar o ícone a ser selecionado, indicado pela seta.

% \begin{figure}[H]
%   \centering
%   \caption{Captura de Tela do Jamovi}
%   \includegraphics[width=\textwidth]{imagens/cap_1/importa_dados_1.png}
%   \label{fig:importa_dados_1}
% \end{figure}

% Após clicar em “Abrir”, selecione a opção “Este PC”. No exemplo apresentado, alguns arquivos que já foram abertos anteriormente são exibidos. Entretanto, se este for o primeiro uso do Jamovi, é necessário clicar em “Procurar” e navegar até o local em que o arquivo de interesse está salvo para, então, importá-lo no software. Para localizar o arquivo, basta navegar pelas pastas até encontrar o diretório em que o arquivo está armazenado.

% \begin{figure}[H]
%   \centering
%   \caption{Captura de Tela do Jamovi}
%   \includegraphics[width=\textwidth]{imagens/cap_1/importa_dados_2.png}
%   \label{fig:importa_dados_2}
% \end{figure}

% Ao clicar no arquivo que contém o conjunto de dados, o Jamovi realizará automaticamente a configuração e importação dos dados. Se você tiver importado os mesmos dados do exemplo apresentado, uma janela semelhante à imagem abaixo será exibida.

% \begin{figure}[H]
%   \centering
%   \caption{Captura de Tela do Jamovi}
%   \includegraphics[width=\textwidth]{imagens/cap_1/jamovi_arquivo_importado.png}
%   \label{fig:jamovi_arquivo_importado}
% \end{figure}

% Após a importação dos dados, você estará pronto(a) para começar a realizar análises estatísticas e utilizar todas as funcionalidades disponíveis no Jamovi. É importante praticar a importação de diferentes conjuntos de dados para se familiarizar com o processo e estar apto(a) a realizá-lo com facilidade.

% \begin{tcolorbox}[colback=white,colframe=green,title= Dica de Conteúdo]
%   Se você teve dificuldades para fazer a importação dos dados não se preocupe. Essa video aula vai te mostrarcom todos os detalhes como você pode fazer a importação dos dados do Jamovi. Destaco que essa é uma das etapas mais importantes, logo tenha certeza de que entendeu corretamente para que seu aprendizado não fique prejudicado.\\
%   \faYoutube{} \href{https://www.youtube.com/watch?v=NIpt0wIq5pc&t=1s}{Importar Dados no Jamovi}
% \end{tcolorbox}
\chapter{Manipulação de Dados no Jamovi}

O foco deste capítulo é a manipulação de dados, uma habilidade essencial para qualquer pessoa que esteja interessada em análise de dados. Vamos explorar as diversas funções e recursos do Jamovi para tratar, organizar e manipular conjuntos de dados. Se você já se perguntou como criar e gerenciar categorias, transformar dados ou manipular variáveis no Jamovi, este capítulo irá te guiar através desses processos passo a passo.

No capítulo anterior você deu os primeiros passos e apresentamos a você a interface do Jamovi para manipulação de dados. Nesse capítulo discutiremos o fluxo de trabalho ideal para o tratamento de um conjunto de dados. O objetivo é garantir que você tenha uma compreensão sólida das ferramentas disponíveis e de como usá-las eficientemente.

Em seguida, abordaremos como criar novas categorias a partir de variáveis existentes. Isso pode ser útil em uma variedade de contextos, como quando você deseja agrupar respostas de pesquisas ou classificar dados em grupos específicos. Também ensinaremos a transformar variáveis, permitindo que você mude o formato dos seus dados de uma maneira que melhor atenda às suas necessidades analíticas.

Finalmente, traremos exemplos práticos para aplicar o conhecimento adquirido. O intuito é promover a familiarização com as ferramentas do software e reforçar a compreensão das funcionalidades abordadas.

Lembre-se, a manipulação eficaz dos dados é a base de qualquer análise de qualidade. Por isso, este capítulo desempenha um papel fundamental no seu aprendizado sobre o uso do Jamovi. Esperamos que ao final desta etapa, você se sinta confiante para manipular conjuntos de dados e prepará-los para a análise de uma maneira eficiente e eficaz.

\section{Introdução à Manipulação de Dados}
Nesta seção, apresentaremos o papel crucial da manipulação de dados na análise de dados. Exploraremos seu propósito, benefícios e relevância no contexto do software Jamovi.

Deixe-me contar algo importante antes de começarmos a falar sobre análises estatísticas complexas. Sabe aquele momento em que você recebe um conjunto de dados e se pergunta "por onde começo"? Pois é, antes de criar aqueles gráficos impressionantes ou realizar testes estatísticos sofisticados, existe uma etapa fundamental que muitas vezes passa despercebida: a manipulação de dados.

Quando eu comecei a trabalhar com análise de dados, confesso que subestimava essa etapa. Achava que era apenas uma questão de abrir a planilha e começar a analisar. Mas rapidamente aprendi que dados do mundo real raramente vêm organizados e prontos para uso. Eles chegam com erros de digitação, valores faltantes, formatos inconsistentes e tantos outros problemas que, se não tratados adequadamente, podem comprometer toda a análise posterior.

A manipulação de dados é como preparar o terreno antes de construir uma casa. Sem uma base sólida, todo o resto fica comprometido. Ela envolve desde a limpeza básica (corrigir erros, eliminar duplicatas), passando pela transformação de variáveis (converter formatos, criar novas variáveis a partir das existentes), até filtragem, agregação e reestruturação dos dados. Parece trabalhoso? Sim, pode ser. Mas acredite, este investimento inicial economiza horas de frustração mais tarde.

O Jamovi, felizmente, nos ajuda bastante nessa jornada. Ele oferece várias ferramentas que facilitam a manipulação básica e intermediária dos dados. Você pode criar e transformar variáveis, recodificar valores, filtrar observações, tratar dados ausentes e muito mais, tudo através de uma interface amigável que não exige conhecimentos avançados de programação.

No entanto, é importante que você saiba que, como qualquer software, o Jamovi tem suas limitações. Quando trabalhamos com conjuntos de dados muito grandes ou precisamos fazer operações de reestruturação complexas, podemos enfrentar alguns desafios. O software pode ficar lento ao processar milhares de observações, e algumas técnicas avançadas de limpeza de dados não estão disponíveis nativamente.

Por isso, em muitos dos meus projetos, acabo utilizando uma abordagem híbrida. Uso o Excel ou Google Sheets para algumas limpezas iniciais e organização básica, principalmente quando preciso fazer verificações visuais rápidas ou ajustes em massa. Para manipulações mais complexas, recorro ocasionalmente ao R ou Python, especialmente em projetos maiores. Só então importo os dados para o Jamovi para a análise estatística final.

Não se preocupe se você não tem experiência com essas outras ferramentas. O Jamovi é perfeitamente capaz de lidar com a maioria das situações que você encontrará, especialmente em contextos acadêmicos e pesquisas de pequeno a médio porte. E neste capítulo, vou compartilhar com você algumas dicas e truques que aprendi ao longo dos anos para contornar as limitações do software.

O mais importante é entender que a manipulação de dados não é apenas uma etapa técnica, mas uma parte fundamental do processo analítico que influencia diretamente a qualidade dos seus resultados. Quanto mais limpos e bem organizados estiverem seus dados, mais confiáveis serão suas análises e conclusões.

Nas próximas páginas, vamos explorar juntos como realizar essas tarefas no Jamovi, com exemplos práticos e orientações passo a passo. Meu objetivo é que, ao final deste capítulo, você se sinta confiante para transformar aqueles dados brutos e desorganizados em um conjunto pronto para revelar seus segredos através da análise estatística.

\section{Interface do Jamovi}
Este segmento é dedicado a apresentar a interface de manipulação de dados do Jamovi. Discutiremos os componentes visuais e explicaremos como interagir com eles para realizar diversas operações.

\section{Importação de Dados}
Aqui, abordaremos o processo de importação de dados no Jamovi. Enfatizaremos a importância de compreender diferentes formatos de dados e como manuseá-los no software.

\section{Criação de Variáveis}
Nesta seção, ensinaremos como criar novas variáveis a partir de variáveis já existentes. Essa é uma habilidade fundamental ao trabalhar com grandes conjuntos de dados.

\section{Transformação de Variáveis}
A transformação de variáveis é um aspecto essencial da manipulação de dados. Aqui, exploraremos as diferentes maneiras de transformar variáveis no Jamovi.

\section{Criação de Categorias}
Nessa seção, discutiremos como criar e gerenciar categorias no Jamovi. Isso é particularmente útil ao lidar com variáveis categóricas e dados de pesquisa.

Para ilustrar como isso é feito, usaremos o \href{https://github.com/balaio-cientifico/dataset/blob/main/pop_ts.txt}{Dataset de População - IBGE} como exemplo. Este dataset contém dados sobre a população dos municípios brasileiros, com informações coletadas nos censos de 2010 e 2022, bem como projeções populacionais realizadas pelo IBGE. Os dados de população são especialmente adequados para a criação de categorias, já que podem ser agrupados de várias maneiras.

Neste tutorial, iremos focar em como usar esse dataset para criar categorias populacionais que representam cidades grandes, médias e pequenas. Este é um exemplo comum de como os dados populacionais são categorizados para análises demográficas, urbanísticas ou sociais.

Para começar, precisamos definir quais são os critérios para uma cidade ser considerada grande, média ou pequena. Essa definição pode variar dependendo do contexto, mas para os fins deste tutorial, iremos definir as categorias da seguinte forma: cidades pequenas são aquelas com população inferior a 20.000 habitantes, cidades médias possuem entre 20.000 e 100.000 habitantes, e cidades grandes são aquelas com mais de 100.000 habitantes.\footnote{Por favor, note que essa classificação é puramente ilustrativa e serve apenas para o propósito deste tutorial. Essa divisão de categorias não possui uma base científica rigorosa e pode variar consideravelmente dependendo do contexto específico. É importante ressaltar que, em sua própria análise, você está livre para definir suas próprias categorias com base nos critérios que considerar mais relevantes para o seu estudo.}

Após a importação dos dados, o primeiro passo para criar categorias no Jamovi é utilizar a opção "Transformar Variáveis". Existem duas maneiras de acessar essa funcionalidade.

A primeira maneira é através do menu de dados. Clique no menu "Dados" na parte superior da interface do Jamovi. Em seguida, selecione a opção "Transformar". Neste ponto, você pode selecionar a variável que deseja transformar em categorias.

A segunda opção, que é um pouco mais direta, envolve clicar com o botão direito do mouse no cabeçalho da variável que você deseja transformar. No menu que aparecerá, selecione a opção "Transformar".

Essas ações são ilustradas na Figura~\ref{fig:criar_categoria_jamovi}. Essa figura indica onde você precisa clicar para acessar a opção de transformação de variáveis.

Por favor, note que a escolha do método para acessar a opção de transformação de variáveis depende da sua preferência. Ambos os métodos levam ao mesmo resultado, então você pode escolher o que achar mais conveniente.

\begin{figure}[H]
    \centering
    \caption{Selecionando a Opção Transformar no Jamovi}
    \includegraphics[width=\textwidth]{imagens/cap_2/criar_categoria_jamovi.png}
    \label{fig:criar_categoria_jamovi}
\end{figure}

No exemplo que vamos tratar neste tutorial, estaremos utilizando a variável \textit{pop\_ibge\_2022}. Esta variável refere-se à população dos municípios brasileiros conforme estimado no censo do IBGE de 2022. Esta variável contínua será transformada em uma variável categórica que representa cidades grandes, médias e pequenas, de acordo com os critérios de classificação que definimos anteriormente.

Agora, precisamos configurar a nova variável categórica que será criada. Nesta etapa, você deverá escolher um nome para a nova variável. No nosso exemplo, chamaremos essa variável de \textit{cat\_pop\_2022}, mas sinta-se livre para escolher o nome que preferir. 

Além disso, é uma boa prática incluir uma descrição para a variável, que explique brevemente o que ela representa. No nosso caso, a descrição será ``Categoria do tamanho das cidades''. Mais uma vez, sinta-se à vontade para criar uma descrição que se adeque ao seu contexto.

É importante confirmar que a variável alvo, neste caso \textit{pop\_ibge\_2022}, foi corretamente selecionada, como é mostrado na Figura~\ref{fig:criar_categoria_jamovi_2}. Essa verificação ajuda a evitar erros na transformação dos dados.

\begin{figure}[H]
    \centering
    \caption{Configurando a nova variável categoórica}
    \includegraphics[width=\textwidth]{imagens/cap_2/criar_categoria_jamovi_2.png}
    \label{fig:criar_categoria_jamovi_2}
\end{figure}

Conforme pode ser observado na Figura \ref{fig:criar_categoria_jamovi_3}, a nova variável foi criada e agora precisamos configurar a transformação que será aplicada. Para isso, basta clicar no botão para adicionar uma nova transformação.

É importante salientar que estamos considerando, para fins deste tutorial, que você nunca realizou uma transformação desse tipo antes. Portanto, se este for o seu caso, não se preocupe, pois todas as etapas serão explicadas detalhadamente para auxiliá-lo(a) neste processo.

\begin{figure}[H]
    \centering
    \caption{Nova variável categórica criada}
    \includegraphics[width=\textwidth]{imagens/cap_2/criar_categoria_jamovi_3.png}
    \label{fig:criar_categoria_jamovi_3}
\end{figure}

Neste ponto, você deverá especificar um nome para a configuração da transformação. Esta etapa é similar à definição de uma função em programação - o Jamovi irá armazenar esta configuração que poderá ser usada posteriormente.

A Figura~\ref{fig:criar_categoria_jamovi_4} ilustra esta etapa. O primeiro campo é onde você insere o nome da transformação. Em seguida, você tem a opção de adicionar uma descrição. Embora isso seja opcional, sempre aconselho a preencher este campo para ajudar a lembrar o que a transformação faz, especialmente se você planeja compartilhar seu trabalho com outras pessoas ou se estiver trabalhando em um projeto de longo prazo.

Finalmente, o passo 3 é onde você seleciona a opção "Adicionar condição de recodificação". Esta é a etapa onde você irá definir as múltiplas condições que irão criar a sua variável categórica.

\begin{figure}[H]
    \centering
    \caption{Configura as categorias para criação das categorias}
    \includegraphics[width=\textwidth]{imagens/cap_2/criar_categoria_jamovi_4.png}
    \label{fig:criar_categoria_jamovi_4}
\end{figure}

Chegamos agora à fase em que iremos definir as condições para a criação da nossa variável categórica no Jamovi. Para fazer isso, precisaremos escrever expressões lógicas, um recurso muito comum em programação. 

Como dito anteriormente, o Jamovi foi construído com base na linguagem R, o que significa que muitos dos recursos de sintaxe do R são aplicáveis ao Jamovi. Portanto, uma compreensão básica da linguagem R pode ser muito útil ao usar o Jamovi.

Se você se sentir um pouco perdido(a) nesse ponto, não se preocupe! Para aqueles que desejam se aprofundar mais no uso da linguagem R e, por consequência, melhorar suas habilidades no Jamovi, recomendo o meu livro de \href{https://www.amazon.com.br/Fundamentos-Completo-Iniciantes-programa%C3%A7%C3%A3o-computa%C3%A7%C3%A3o-ebook/dp/B0B36NG18N}{Fundamentos em R: Guia Completo para Iniciantes}. Nele, eu exploro em detalhes como utilizar o R, o que pode ser de grande ajuda para aprimorar seu domínio do Jamovi.

Veja na Figura~\ref{fig:criar_categoria_jamovi_5}, o local em que você deve selecionar as condições para criarmos as categorias de cidades: pequenas, médias e grandes. Clique na seção de operadores e selecione ou escreva conforme eu escrevi na tabela \ref{fig:criar_categoria_jamovi_6}

\begin{figure}[H]
    \centering
    \caption{Seleciona a caixa para escrever as condições no Jamovi}
    \includegraphics[width=\textwidth]{imagens/cap_2/criar_categoria_jamovi_5.png}
    \label{fig:criar_categoria_jamovi_5}
\end{figure}

Vamos entender a lógica que o Jamovi utiliza para criar categorias. O Jamovi avalia as condições em sequência, considerando a condição anterior.  Por exemplo, ao definir a categoria "Pequena" para cidades com população menor ou igual a 20.000 habitantes, inserimos a expressão $\leqslant 20000$. Em seguida, para classificar as cidades de tamanho "Médio", utilizamos a expressão $\leqslant 100000$. Neste momento, o Jamovi, automaticamente, entende que esta condição se refere à população que varia de 20.001 a 100.000 habitantes, já que a classificação anterior já considerou as cidades com população menor ou igual a 20.000.

Na última condição, vamos utilizar o operador ELSE para definir as cidades grandes, pois agora só restam elas a serem classificadas. 

A Figura \ref{fig:criar_categoria_jamovi_6} mostra os 4 passos. Os três primeiros passos são as definições das condições de transformação e a quarta etapa é a verificação.

Um recurso interessante do Jamovi é que, enquanto você configura as transformações, o software automaticamente começa a criar as categorias para você, permitindo visualizar em tempo real se a recodificação está ocorrendo corretamente. 

\begin{figure}[H]
    \centering
    \caption{Criando as condições para Transformação de Variáveis}
    \includegraphics[width=\textwidth]{imagens/cap_2/criar_categoria_jamovi_6.png}
    \label{fig:criar_categoria_jamovi_6}
\end{figure}

Em nosso exemplo, nós escrevemos os nomes ``Pequena'', ``Média'' e ``Grande'' diretamente entre aspas, pois, relembrando, o Jamovi possui muitos componentes herdados da linguagem R. A linguagem R, assim como muitas outras linguagens de programação, usa aspas para designar textos. Portanto, sempre que quisermos definir uma categoria com um nome de texto, devemos colocar esse nome entre aspas.

Se você quiser criar categorias com códigos numéricos, as etapas são as mesmas, mas você deve utilizar números sem aspas. Por exemplo, poderíamos codificar ``Pequena'' como 1, ``Média'' como 2 e ``Grande'' como 3. Nesse caso, ao invés de escrever ``Pequena'', ``Média'' e ``Grande'' nas condições, escreveríamos os números correspondentes.

É importante lembrar que, ao usar códigos numéricos, será necessário criar uma legenda para lembrar o que cada número significa. Jamovi permite que você faça isso através de ``Níveis de Variáveis''.

Com a definição de todas as condições, a variável categórica está pronta para ser utilizada em suas análises. Como você pode ver, o processo de transformar uma variável contínua em uma variável categórica é bastante simples e intuitivo no Jamovi.

Para consolidar esse aprendizado, sugerimos que pratique o processo de criação de categorias com outros conjuntos de dados e exemplos. Lembre-se, a prática é uma parte importante do aprendizado. Quanto mais você praticar, mais natural o processo se tornará.

\section{Tratamento de Dados Faltantes}
Dados faltantes ou incompletos são uma realidade em análise de dados. Nesta seção, ensinaremos como identificar e tratar esses dados no Jamovi.

\section{Filtragem de Dados}
A filtragem é um passo crucial na preparação dos dados. Aqui, mostraremos como selecionar e filtrar subconjuntos de dados no Jamovi.

\section{Ordenação de Dados}
Nesta seção, abordaremos a ordenação de dados, uma operação importante para a visualização eficaz e análise de conjuntos de dados.

\section{Renomear e Reorganizar Variáveis}
Aqui, falaremos sobre como renomear e reorganizar variáveis no Jamovi. Esses são passos fundamentais na organização dos dados para análise.

\section{Dicas e Truques}
Esta seção apresentará dicas e truques úteis para melhorar a eficiência na manipulação de dados no Jamovi. Buscaremos compartilhar os melhores métodos e práticas recomendadas.

\section{Resumo do Capítulo}
Concluiremos o capítulo com um resumo, relembrando os principais tópicos abordados e reforçando os conceitos chave aprendidos sobre manipulação de dados no Jamovi.

\chapter{Estatística Descritiva}

Neste capítulo da apostila, vamos explorar o passo a passo de como realizar análises de estatística descritiva no Jamovi. Aprenderemos como utilizar as diversas ferramentas disponíveis no software para organizar, resumir e apresentar os dados de maneira clara e concisa.

No capítulo anterior nós aprendemos como importar os dados para o Jamovi, seja através da importação de arquivos ou inserção direta na planilha. Em seguida, abordaremos as principais medidas de tendência central, como média, mediana e moda, e as medidas de dispersão, como desvio padrão, variância e amplitude. Você aprenderá como calcular essas medidas usando o Jamovi e interpretar os resultados.

Em seguida, mergulharemos nos recursos de visualização de dados do Jamovi. Exploraremos os diversos gráficos disponíveis, como gráficos de barras, gráficos de dispersão, gráficos de linha, gráficos de boxplot e histogramas. Você aprenderá como criar esses gráficos, personalizá-los e interpretar as informações que eles fornecem sobre seus dados.

Continuaremos com a criação de tabelas de frequência, que mostrarão a distribuição dos dados em categorias ou intervalos. Você aprenderá a criar tabelas de frequência no Jamovi e interpretar os resultados para entender a distribuição dos seus dados.

Em seguida, exploraremos as tabelas cruzadas, uma técnica poderosa para analisar a relação entre duas ou mais variáveis. Aprenderemos como criar tabelas cruzadas no Jamovi e interpretar os resultados para identificar associações ou padrões entre as variáveis.

Por fim, abordaremos a criação de relatórios de estatísticas descritivas no Jamovi. Você aprenderá como criar relatórios que resumem suas análises, incluindo informações sobre as medidas calculadas, gráficos relevantes e interpretação dos resultados. Esses relatórios serão úteis para compartilhar seus resultados com outros pesquisadores ou colaboradores.

Ao final deste capítulo, você terá adquirido as habilidades necessárias para realizar análises de estatística descritiva no Jamovi. Você estará apto a utilizar as diversas ferramentas e recursos oferecidos pelo software para explorar e descrever seus dados de forma eficiente e precisa. A estatística descritiva no Jamovi será uma poderosa aliada em suas análises e na comunicação clara dos resultados obtidos.

\begin{tcolorbox}[colback=white,colframe=green,title= Dica de Conteúdo]
    Essa apostila não é um livro de estatística. Ela é um guia de como utilizar o Jamovi para realizar análises estatísticas. Se você não conhece os conceitos estatísticos, recomendo que você estude os conceitos antes de utilizar o Jamovi. Nessa apostila eu cito algumas bibliografias que podem te ajudar a entender os conceitos estatísticos. Visite a parte de Bibliografia e leia os livros indicados caso você julgue necessário. Entretanto, durante as sessões eu faço uma breve introdução aos conceitos estatísticos para que você possa entender o que está sendo feito no Jamovi.
\end{tcolorbox}

A estatística descritiva é uma parte fundamental da análise de dados, pois oferece uma visão geral e resumida das características dos dados. No Jamovi, um software estatístico de código aberto e amigável, você pode encontrar várias ferramentas e recursos para realizar estatística descritiva de maneira eficiente.

Ao utilizar o Jamovi, você pode importar seus dados ou inseri-los diretamente na planilha. Em seguida, você pode explorar as diversas opções disponíveis para realizar análises descritivas. Alguns dos recursos mais comuns incluem:

\begin{itemize}
    \item Medidas de tendência central: O Jamovi oferece várias opções para calcular medidas de tendência central, como média, mediana e moda. Essas medidas ajudam a identificar valores centrais ou típicos em seus dados.
    \item Medidas de dispersão: Além das medidas de tendência central, o Jamovi também permite calcular medidas de dispersão, como desvio padrão, variância e amplitude. Essas medidas fornecem informações sobre a variação ou dispersão dos seus dados.
    \item Gráficos: O Jamovi possui uma ampla variedade de gráficos disponíveis para visualizar seus dados de forma clara e compreensível. Você pode criar gráficos de barras, gráficos de dispersão, gráficos de linha, gráficos de boxplot, histogramas, entre outros. Esses gráficos podem ajudar a identificar padrões, tendências e outliers nos seus dados.
    \item Tabelas de frequência: O Jamovi permite criar tabelas de frequência, que mostram a distribuição dos seus dados em categorias ou intervalos. Essas tabelas são úteis para identificar a frequência de ocorrência de valores específicos ou a distribuição dos seus dados em diferentes categorias.
    \item Tabelas cruzadas: Com o Jamovi, você pode criar tabelas cruzadas para explorar a relação entre duas ou mais variáveis. Isso permite analisar como as diferentes variáveis estão relacionadas entre si e identificar possíveis associações ou padrões.
    \item Relatórios: O Jamovi oferece a opção de criar relatórios de estatísticas descritivas para compartilhar com outros pesquisadores ou colaboradores. Esses relatórios podem incluir informações sobre as análises realizadas, resultados obtidos e gráficos relevantes, facilitando a comunicação dos resultados de forma clara e concisa.
\end{itemize}

Além desses recursos, o Jamovi também oferece suporte a uma ampla gama de técnicas estatísticas avançadas, como testes de hipóteses, análise de variância (ANOVA), regressão, análise fatorial e muito mais. Com sua interface intuitiva e recursos poderosos, o Jamovi é uma ferramenta acessível e eficiente para realizar análises estatísticas descritivas.

\section{Medidas de tendência central}

As medidas de tendência central são estatísticas utilizadas na análise descritiva para resumir e descrever um conjunto de dados, fornecendo uma medida representativa do centro dos valores observados. Essas medidas são úteis para entender as características e propriedades centrais dos dados, permitindo uma compreensão mais clara e concisa da distribuição dos valores \parencite{Triola2017}.

\subsection{Média}

A média é uma medida de tendência central amplamente utilizada na estatística descritiva. É calculada como a soma de todos os valores em um conjunto de dados dividida pelo número de observações. A média é frequentemente utilizada para determinar um valor típico ou representativo de um conjunto de dados.

A média é uma medida robusta, pois leva em consideração todos os valores do conjunto de dados. Ela é especialmente útil quando os dados estão distribuídos de forma relativamente simétrica e não possuem valores extremos significativos. No entanto, é importante ter cuidado ao usar a média quando há valores discrepantes ou uma distribuição assimétrica, pois esses casos podem distorcer a interpretação dos resultados \parencite{Triola2017}.

A média é fácil de calcular e fornece uma representação numérica única para resumir os dados. Ela possui propriedades matemáticas úteis, como a propriedade de preservar a soma (a soma das médias é igual à média das somas) e pode ser usada para comparar conjuntos de dados diferentes.

\[
\bar{x} = \frac{1}{n} \sum_{i=1}^{n} x_i
\]

Ao interpretar a média, é importante considerar o contexto do problema em questão. Ela pode ser utilizada para compreender características centrais de um conjunto de dados, como a média de salários de uma população, a média de notas de um grupo de estudantes ou a média de idades de um determinado grupo.

No entanto, é fundamental lembrar que a média não fornece informações sobre a dispersão ou variabilidade dos dados. Portanto, é sempre recomendado complementar a análise com outras medidas descritivas, como o desvio padrão, para ter uma compreensão completa da distribuição dos dados.

Para calcular a média no Jamovi, você pode seguir as etapas abaixo:

\begin{enumerate}
    \item Importe seus dados ou insira-os diretamente na planilha do Jamovi. Certifique-se de que os dados estejam organizados em uma única coluna ou em várias colunas, dependendo da sua estrutura.
    \item Selecione a coluna que contém os dados dos quais você deseja calcular a média. Para fazer isso, clique no cabeçalho da coluna ou arraste o mouse para selecionar várias colunas.
    \item No painel de Análises, localizado à direita da tela, clique na seção "Descriptive Statistics" (Estatísticas Descritivas).
    \item Na seção de estatísticas descritivas, você encontrará várias opções. Procure por "Mean" (Média) e marque a caixa ao lado dela.
    \item Assim que você marcar a caixa de seleção "Mean" (Média), o Jamovi calculará automaticamente a média para os dados selecionados. O valor da média será exibido na tabela de resultados.
    \item Se você deseja realizar o cálculo da média para grupos específicos de dados, você pode utilizar a função de agrupamento. Para isso, clique no ícone "Group by" (Agrupar por) no painel de estatísticas descritivas. Selecione a variável que contém os grupos ou categorias e o Jamovi calculará as médias separadamente para cada grupo.
    \item Para visualizar os resultados de forma mais clara, você pode criar um gráfico de barras ou um gráfico de linha que exiba as médias para cada grupo. Para isso, clique na seção "Visualize" (Visualizar) no painel de estatísticas descritivas e escolha o tipo de gráfico que melhor se adequa aos seus dados.
\end{enumerate}

\subsection{Moda}

A moda é uma medida de tendência central na estatística descritiva que representa o valor ou valores que ocorrem com maior frequência em um conjunto de dados. Em outras palavras, a moda indica o valor mais comum ou popular em um conjunto de observações \parencite{Triola2017}.

\[
\text{moda} = \text{valor mais frequente na amostra}
\]

No conjunto de dados representado dentro do círculo abaixo (\ref{fig:circulo_moda}), cada número é desenhado em uma posição aleatória dentro do círculo. Analisando os números desenhados, podemos observar que o número 6 ocorre duas vezes, mais do que qualquer outro número no conjunto. Portanto, podemos concluir que o número 6 é a moda desse conjunto de dados.

\begin{figure}[H]
    \centering
    \caption{Exemplo do que é moda}
    \begin{tikzpicture}[scale=1.5]
        % Define o círculo
        \draw (0,0) circle (2cm);

        % Define os números
        \def\numbers{{1, 2, 3, 4, 5, 6, 6, 7, 8, 9, 10}}
        
        % Define a distância mínima entre os números
        \def\minDistance{0.4} % Ajuste o valor conforme necessário
        
        % Define o raio máximo permitido
        \def\maxRadius{1.6} % Ajuste o valor conforme necessário

        % Desenha os números
        \foreach \i in {1,...,11} {
            % Obtém o número atual
            \pgfmathparse{\numbers[\i-1]}
            \let\number\pgfmathresult
            
            % Define um ângulo e um raio aleatórios
            \pgfmathsetmacro{\angle}{rnd*360}
            \pgfmathsetmacro{\radius}{\minDistance + rnd*(\maxRadius-\minDistance)}
            
            % Verifica se algum número já ocupa a posição
            \pgfmathtruncatemacro{\overlap}{0}
            \foreach \j in {1,...,\i} {
                \pgfmathparse{\numbers[\j-1]}
                \let\existingNumber\pgfmathresult
                \pgfmathparse{atan2(sin(\angle)*\radius-sin(\numbers[\j-1])*2,cos(\angle)*\radius-cos(\numbers[\j-1])*2)}
                \pgfmathtruncatemacro{\angleDiff}{\pgfmathresult}
                \ifnum\angleDiff=0
                    \ifnum\number=\existingNumber
                        \pgfmathtruncatemacro{\overlap}{1}
                    \fi
                \fi
                \ifnum\angleDiff=0
                    \ifdim\radius pt<\minDistance pt
                        \pgfmathtruncatemacro{\overlap}{1}
                    \fi
                \fi
            }
            
            % Verifica se o número é igual a 6 e não há sobreposição
            \ifnum\number=6
                \ifnum\overlap=0
                    % Se for igual a 6 e não houver sobreposição, desenha o número em vermelho
                    \node[text=red, circle, minimum size=0.3cm] at (\angle:\radius cm) {\number};
                \else
                    % Se houver sobreposição, não desenha o número
                \fi
            \else
                % Caso contrário, desenha o número normalmente
                \ifnum\overlap=0
                    \node[circle, minimum size=0.3cm] at (\angle:\radius cm) {\number};
                \else
                    % Se houver sobreposição, não desenha o número
                \fi
            \fi
        }
    \end{tikzpicture}
    \label{fig:circulo_moda}
\end{figure}

Ao contrário da média, que é calculada a partir da soma de todos os valores dividida pelo número de observações, a moda é determinada pela identificação do valor ou valores que aparecem com maior frequência. Pode haver um único valor de moda, conhecido como moda unimodal, ou pode haver mais de um valor de moda, chamado de moda bimodal, trimodal ou multimodal.

A moda é especialmente útil quando se lida com dados categóricos ou discretos, como categorias de produtos, cores favoritas, resultados de votações, entre outros. Ela também pode ser aplicada a dados contínuos, embora seja menos comum nesse contexto.

A moda é uma medida relativamente fácil de calcular, uma vez que envolve apenas a contagem dos valores e a identificação do valor mais frequente. No entanto, assim como a média, a moda pode não ser suficiente para descrever completamente a distribuição dos dados. Por exemplo, em um conjunto de dados em que todos os valores ocorrem apenas uma vez, não haverá um valor de moda claro.

É importante observar que, ao contrário da média e da mediana, a moda não é afetada por valores extremos ou discrepantes, pois seu cálculo é baseado na frequência de ocorrência dos valores. Isso faz com que a moda seja uma medida robusta em relação a valores atípicos.

No Jamovi, calcular a moda pode ser feito seguindo este passo a passo:

\begin{enumerate}
    \item Importe seus dados ou insira-os diretamente na planilha do Jamovi. Certifique-se de que os dados estejam organizados em uma única coluna ou em várias colunas, dependendo da sua estrutura.
    \item Selecione a coluna que contém os dados dos quais você deseja calcular a moda. Para fazer isso, clique no cabeçalho da coluna ou arraste o mouse para selecionar várias colunas.
    \item No painel de Análises, localizado à direita da tela, clique na seção "Descriptive Statistics" (Estatísticas Descritivas).
    \item Na seção de estatísticas descritivas, você encontrará várias opções. Procure por "Mode" (Moda) e marque a caixa ao lado dela.
    \item Assim que você marcar a caixa de seleção "Mode" (Moda), o Jamovi calculará automaticamente a moda para os dados selecionados. O valor ou valores da moda serão exibidos na tabela de resultados.
    \item Se você deseja calcular a moda para grupos específicos de dados, utilize a função de agrupamento. Clique no ícone "Group by" (Agrupar por) no painel de estatísticas descritivas. Selecione a variável que contém os grupos ou categorias, e o Jamovi calculará as modas separadamente para cada grupo.
    \item Caso o conjunto de dados não apresente um valor de moda claro (todos os valores ocorrem apenas uma vez ou não há valores que se repitam com maior frequência), o Jamovi indicará a ausência de moda na tabela de resultados.
\end{enumerate}

\subsection{Mediana}

Suponha que temos a seguinte lista de números: $2, 12, 5, 10, 8, 7, 5$. Para encontrar a mediana, primeiro precisamos ordenar a lista em ordem crescente: $2, 3, 5, 7, 8, 10, 12$. Em seguida, encontramos o valor do meio da lista, que é $7$. Portanto, a mediana desta lista é $7$.

Podemos representar a mediana graficamente usando um desenho. O valor da mediana é o ponto em qvermelho, onde nós dividimos o conjunto de dados em duas partes iguais. Veja o exemplo abaixo:

\begin{center}
    \begin{tikzpicture}
      % Define the boxes
      \draw (0,0) rectangle (1,1);
      \draw (1.5,0) rectangle (2.5,1);
      \draw (3,0) rectangle (4,1);
      \draw [red] (4.5,0) rectangle (5.5,1);
      \draw (6,0) rectangle (7,1) node[midway] {7};
      \draw (7.5,0) rectangle (8.5,1);
      \draw (9,0) rectangle (10,1);
      
      % Label the boxes
      \node at (0.5,0.5) {2};
      \node at (2,0.5) {3};
      \node at (3.5,0.5) {5};
      \node at (5,0.5) {7};
      \node at (6.5,0.5) {8};
      \node at (8,0.5) {10};
      \node at (9.5,0.5) {12};
      
      % Label the median
      \node[red] at (5.0,-0.30) {Mediana};
    \end{tikzpicture} 
\end{center}
    

Neste exemplo, a mediana é $7$, que é o ponto em que a curva do histograma é dividida em duas partes iguais.

Note que, se a lista tivesse um número par de elementos, a mediana seria a média dos dois valores do meio. Por exemplo, se a lista fosse $2, 3, 5, 7, 8, 10$, a mediana seria $(5+7)/2 = 6$.

\section{Medidas de Dispersão}

As medidas de dispersão são estatísticas utilizadas na análise descritiva para quantificar a variabilidade, espalhamento ou dispersão dos valores em um conjunto de dados. Essas medidas fornecem informações sobre o quão dispersos ou concentrados os dados estão em relação a uma medida de tendência central, como a média.

\subsection{Desvio Padrão}

O desvio padrão é uma medida estatística que indica a dispersão ou variabilidade dos valores em relação à média de um conjunto de dados. Ele fornece uma medida da diferença média entre cada valor e a média do conjunto de dados \parencite{Triola2017}.

\begin{figure}[H]
    \centering
    \begin{tikzpicture}[>=latex]
        % Eixo
        \draw[->] (-4,0) -- (4,0) node[right] {$x$};
        \draw[->] (0,-0.5) -- (0,3) node[above] {Frequência};
    
        % Pontos de dados
        \foreach \x/\y in {-2/0.5, -1/1.5, 0/2, 1/1, 2/0.5} {
        \fill (\x,\y) circle (2pt);
        }
    
        % Linhas verticais
        \foreach \x/\y in {-2/0.5, -1/1.5, 0/2, 1/1, 2/0.5} {
        \draw[dashed] (\x,\y) -- (\x,0);
        }
    
        % Média
        \draw[dashed, red] (0,0) -- (0,2.5) node[right, red] {$\bar{x}$};
    
        % Desvio Padrão
        \draw[<->] (-1.5,-0.3) -- (1.5,-0.3) node[midway, below] {Desvio Padrão};
    
        % Linhas de Desvio
        \draw[blue, <->] (-1,0) -- (-1,1.5) node[midway, left] {$\sigma$};
        \draw[blue, <->] (1,0) -- (1,1) node[midway, right] {$\sigma$};

         % Legenda
         \node at (0,-1.5) {\begin{tabular}{@{}c@{ }l@{}}
            \tikz\draw[blue] (0,0) -- (0.5,0); & $\sigma$ - Desvio padrão\\
            \tikz\draw[red, dashed] (0,0) -- (0.5,0); & $\bar{x}$ - Média
            \end{tabular}};
    
    \end{tikzpicture}
\end{figure}

O desvio padrão é calculado em duas etapas principais: primeiro, calcula-se a diferença entre cada valor e a média; em seguida, essas diferenças são somadas, elevadas ao quadrado, e a média desses quadrados é calculada. A raiz quadrada dessa média é o desvio padrão.

\[
s = \sqrt{\frac{1}{n-1} \sum_{i=1}^{n} (x_i - \bar{x})^2}
\]

Uma interpretação do desvio padrão é que ele mede o quanto os valores se afastam, em média, da média. Quanto maior o desvio padrão, maior é a dispersão dos valores em relação à média. Por outro lado, um desvio padrão menor indica que os valores estão mais próximos da média.

O desvio padrão é uma medida importante para entender a variabilidade e a consistência dos dados. Ele pode ajudar a identificar se os valores estão agrupados ou se estão mais dispersos. Além disso, o desvio padrão é usado em muitas outras técnicas estatísticas, como a inferência estatística, para avaliar a precisão das estimativas.

É importante lembrar que o desvio padrão é sensível a valores extremos, pois eles podem influenciar significativamente a medida. Portanto, é essencial considerar o contexto dos dados e interpretar o desvio padrão em conjunto com outras medidas descritivas, como a média e a mediana, para obter uma visão completa da distribuição dos dados.

No Jamovi, calcular o desvio padrão pode ser feito seguindo este passo a passo:

\begin{enumerate}
    \item Importe seus dados ou insira-os diretamente na planilha do Jamovi. Certifique-se de que os dados estejam organizados em uma única coluna ou em várias colunas, dependendo da sua estrutura.
    \item Selecione a coluna que contém os dados dos quais você deseja calcular o desvio padrão. Para fazer isso, clique no cabeçalho da coluna ou arraste o mouse para selecionar várias colunas.
    \item No painel de Análises, localizado à direita da tela, clique na seção "Descriptive Statistics" (Estatísticas Descritivas).
    \item Na seção de estatísticas descritivas, você encontrará várias opções. Procure por "Standard Deviation" (Desvio Padrão) e marque a caixa ao lado dela.
    \item Assim que você marcar a caixa de seleção "Standard Deviation" (Desvio Padrão), o Jamovi calculará automaticamente o desvio padrão para os dados selecionados. O valor do desvio padrão será exibido na tabela de resultados.
    \item Se você deseja calcular o desvio padrão para grupos específicos de dados, utilize a função de agrupamento. Clique no ícone "Group by" (Agrupar por) no painel de estatísticas descritivas. Selecione a variável que contém os grupos ou categorias, e o Jamovi calculará os desvios padrão separadamente para cada grupo.
    \item O Jamovi também oferece a opção de calcular outros tipos de desvio padrão, como o desvio padrão populacional. Para isso, clique no ícone de configurações ao lado da opção "Standard Deviation" (Desvio Padrão) e selecione a opção desejada.
\end{enumerate}
\input{capitulos/capitulo_4.tex}
\input{capitulos/capitulo_5.tex}
\input{capitulos/capitulo_6.tex}
\input{capitulos/capitulo_7.tex}

\printbibliography

% Define o final do documento
\backmatter

\end{document}